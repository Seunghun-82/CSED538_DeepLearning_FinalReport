\documentclass[../main.tex]{subfiles}
\graphicspath{{\subfix{../images/}}}
\begin{document}

\textbf{}
In recent years, artificial satellites have been employed across various domains such as meteorological observation, military reconnaissance, and have significantly impacted many aspects of human life. Particularly, satellite-captured images offer diverse utility in observing real-time changes occurring on Earth. However, manually identifying these changes by experts results in inefficiencies. Consequently, there has been a growing demand for research identifying specific changes within  sequences of images, which leads to Change Detection task in Computer Vision.

Change Detection involves a computer vision task aimed at identifying differences generally between two images. Its primary objective is to detect significant regional changes, such as new constructions of building or road networks, and even achieve localization in images. The two sequential images are referred to as bi-temporal images. The difficulty in detecting differences between bi-temporal images arises from the influence of the surrounding environment at the time of image capture, which act as noise in images. Indeed, the presence of noise can lead to the extraction of features from images that include the noise, consequently , this might impact the overall performance of the model. To address this issue, one of the recent State-of-the-Art (SOTA) model, DDPM-cd, utilizes a Diffusion model to effectively extract features from images.\cite{bandara2022ddpmcd} It generates distinctive features for each image using the Diffusion model. Subsequently, these generated features are input into a Change Detection Network to identify whether differences exist between the images or not. This methodology allows for improved feature extraction and the identification of changes despite the presence of noise in the images.
\end{document}